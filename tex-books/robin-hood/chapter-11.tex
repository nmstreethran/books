\chapter{How Robin Hood Fought Guy of Gisbourne}

\begin{quote}
“I dwell by dale and down,” quoth he,
“And Robin to take I’m sworn;
And when I am called by my right name,
I am Guy of good Gisborne.”
\end{quote}

\lettrine{S}{ome} weeks passed after the rescue of the widow's three sons; weeks
spent by the Sheriff in the vain effort to entrap Robin Hood and his
men. For Robin's name and deeds had come to the King's ears, in London
town, and he sent word to the Sheriff to capture the outlaw, under
penalty of losing his office. So the Sheriff tried every manner of means
to surprise Robin Hood in the forest, but always without success. And he
increased the price put upon Robin's head, in the hope that the best men
of the kingdom could be induced to try their skill at a capture.

Now there was a certain Guy of Gisborne, a hireling knight of the King's
army, who heard of Robin and of the price upon his head. Sir Guy was one
of the best men at the bow and the sword in all the King's service. But
his heart was black and treacherous. He obtained the King's leave
forthwith to seek out the forester; and armed with the King's scroll he
came before the Sheriff at Nottingham.

``I have come to capture Robin Hood,'' quoth he, ``and mean to have him,
dead or alive.''

``Right gladly would I aid you,'' answered the Sheriff, ``even if the
King's seal were not sufficient warrant. How many men need you?''

``None,'' replied Sir Guy, ``for I am convinced that forces of men can
never come at the bold robber. I must needs go alone. But do you hold
your men in readiness at Barnesdale, and when you hear a blast from this
silver bugle, come quickly, for I shall have the sly Robin within my
clutches.''

``Very good,'' said the Sheriff. ``Marry, it shall be done.'' And he set
about giving orders, while Guy of Gisborne sallied forth disguised.

Now as luck would have it, Will Scarlet and Little John had gone to
Barnesdale that very day to buy suits of Lincoln green for certain of
the yeomen who had come out at the knees and elbows. But not deeming it
best for both of them to run their necks into a noose, together, they
parted just outside the town, and Will went within the gates, while John
tarried and watched at the brow of the hill on the outside.

Presently whom should he see but this same Will flying madly forth from
the gates again, closely pursued by the Sheriff and threescore men. Over
the moat Will sprang, through the bushes and briars, across the swamp,
over stocks and stones, up the woodland roads in long leaps like a
scared jack rabbit. And after him puffed the Sheriff and his men, their
force scattering out in the flight as one man would tumble head-first
into a ditch, another mire up in the swamp, another trip over a rolling
stone, and still others sit down on the roadside and gasp for wind like
fish out of water.

Little John could not forbear laughing heartily at the scene, though he
knew that `twould be anything but a laughing matter if Will should
stumble. And in truth one man was like to come upon him. It was
William-a-Trent, the best runner among the Sheriff's men. He had come
within twenty feet of Scarlet and was leaping upon him with long bounds
like a greyhound, when John rose up quickly, drew his bow and let fly
one of his fatal shafts. It would have been better for William-a-Trent
to have been abed with sorrow--says the ballad--than to be that day in
the greenwood slade to meet with Little John's arrow. He had run his
last race.

The others halted a moment in consternation, when the shaft came
hurtling down from the hill; but looking up they beheld none save Little
John, and with a cry of fierce joy they turned upon him. Meanwhile Will
Scarlet had reached the brow of the hill and sped down the other side.

``I'll just send one more little message of regret to the Sheriff,''
said Little John, ``before I join Will.''

But this foolhardy deed was his undoing, for just as the arrow left the
string, the good yew bow that had never before failed him snapped in
twain.

``Woe worth, woe worth thee, wicked wood, that ere thou grew on a
tree!'' cursed Little John, and planted his feet resolutely in the earth
resolved to sell the path dearly; for the soldiers were now so close
upon him that he dared not turn.

And a right good account of himself he gave that day, dealing with each
man as he came up according to his merit. And so winded were the
pursuers when they reached the top of the hill that he laid out the
first ten of them right and left with huge blows of his brawny fist.

But if five men can do more than three, a score can overcome one.

A body of archers stood off at a prudent distance and covered Little
John with their arrows.

``Now yield you!'' panted the Sheriff. ``Yield you, Little John, or
Reynold Greenleaf, or whatever else name you carry this day! Yield you,
or some few of these shafts will reach your heart!''

``Marry, my heart has been touched by your words ere now,'' said Little
John; ``and I yield me.''

So the Sheriff's men laid hold of Little John and bound him fast with
many cords, so fearful were they lest he should escape. And the Sheriff
laughed aloud in glee, and thought of how he should avenge his stolen
plate, and determined to make a good day's work of it.

``By the Saints!'' he said, ``you shall be drawn by dale and down, and
hanged high on a hill in Barnesdale this very day.''

``Hang and be hanged!'' retorted the prisoner. ``You may fail of your
purpose if it be Heaven's will.''

Back down the hill and across the moor went the company speedily, for
they feared a rescue. And as they went the stragglers joined them. Here
a man got up feebly out of the ditch and rubbed his pate and fell in
like a chicken with the pip going for its dinner. Yonder came hobbling a
man with a lame ankle, or another with his shins torn by the briars or
another with his jacket all muddy from the marsh. So in truth it was a
tatterdemalion crew that limped and straggled and wandered back into
Barnesdale that day. Yet all were merry, for the Sheriff had promised
them flagons of wine, and moreover they were to hang speedily the
boldest outlaw in England, next to Robin Hood himself.

The gallows was quickly put up and a new rope provided.

``Now up with you!'' commanded the Sheriff, ``and let us see if your
greenwood tricks will avail you to-morrow.''

``I would that I had bold Robin's horn,'' muttered poor John; ``methinks
`tis all up with me even as the Sheriff hath spoken.''

In good sooth the time was dire and pressing. The rope was placed around
the prisoner's neck and the men prepared to haul away.

``Are you ready?'' called the Sheriff. ``One--two--''

But before the ``three'' left his lips the faint sound of a silver bugle
came floating over the hill.

``By my troth, that is Sir Guy of Gisborne's horn,'' quoth the Sheriff;
``and he bade me not to delay answering its summons. He has caught Robin
Hood.''

``Pardon, Excellency,'' said one of his men; ``but if he has caught
Robin Hood, this is a merry day indeed. And let us save this fellow and
build another gallows and hang them both together.''

``That's a brave thought!'' said the Sheriff slapping his knee. ``Take
the rascal down and bind him fast to the gallows-tree against our
return.''

So Little John was made fast to the gallows-tree, while the Sheriff and
all his men who could march or hobble went out to get Robin Hood and
bring him in for the double hanging.

Let us leave talking of Little John and the Sheriff, and see what has
become of Robin Hood.

In the first place, he and Little John had come near having a quarrel
that self-same morning because both had seen a curious looking yeoman,
and each wanted to challenge him singly. But Robin would not give way to
his lieutenant, and that is why John, in a huff, had gone with Will to
Barnesdale.

Meanwhile Robin approached the curious looking stranger. He seemed to be
a three-legged creature at first sight, but on coming nearer you would
have seen that `twas really naught but a poorly clad man, who for a
freak had covered up his rags with a capul-hide, nothing more nor less
than the sun-dried skin of a horse, complete with head, tail, and mane.
The skin of the head made a helmet; while the tail gave the curious
three-legged appearance.

``Good-morrow, good fellow,'' said Robin cheerily, ``methinks by the bow
you bear in your hand that you should be a good archer.''

``Indifferent good,'' said the other returning his greeting; ``but `tis
not of archery that I am thinking this morning, for I have lost my way
and would fain find it again.''

``By my faith, I could have believed `twas your wits you'd lost!''
thought Robin smiling. Then aloud: ``I'll lead you through the wood,''
quoth he, ``an you will tell me your business. For belike your speech is
much gentler than your attire.''

``Who are you to ask me my business?'' asked the other roughly.

``I am one of the King's Rangers,'' replied Robin, ``set here to guard
his deer against curious looking strollers.''

``Curious looking I may be,'' returned the other, ``but no stroller.
Hark ye, since you are a Ranger, I must e'en demand your service. I am
on the King's business and seek an outlaw. Men call him Robin Hood. Are
you one of his men?''--eyeing him keenly.

``Nay, God forbid!'' said Robin; ``but what want you with him?''

``That is another tale. But I'd rather meet with that proud outlaw than
forty good pounds of the King's money.''

Robin now saw how the land lay.

``Come with me, good yeoman,'' said he, ``and belike, a little later in
the day, I can show you Robin's haunts when he is at home. Meanwhile let
us have some pastime under the greenwood tree. Let us first try the
mastery at shooting arrows.''

The other agreed, and they cut down two willow wands of a summer's
growth that grew beneath a brier, and set them up at a distance of
threescore yards.

``Lead on, good fellow,'' quoth Robin. ``The first shot to you.''

``Nay, by my faith,'' said the other, ``I will follow your lead.''

So Robin stepped forth and bent his bow carelessly and sent his shaft
whizzing toward the wand, missing it by a scant inch. He of the
horse-hide followed with more care yet was a good three-fingers' breadth
away. On the second round, the stranger led off and landed cleverly
within the small garland at the top of the wand; but Robin shot far
better and clave the wand itself, clean at the middle.

``A blessing on your heart!'' shouted Capul-Hide; ``never saw I such
shooting as that! Belike you are better than Robin Hood himself. But you
have not yet told me your name.''

``Nay, by my faith,'' quoth Robin, ``I must keep it secret till you have
told me your own.''

``I do not disdain to tell it,'' said the other. ``I dwell by dale and
down, and to take bold Robin am I sworn. This would I tell him to his
face, were he not so great a craven. When I am called by my right name,
I am Guy of Gisborne.''

This he said with a great show of pride, and he strutted back and forth,
forgetful that he had just been beaten at archery.

Robin eyed him quietly. ``Methinks I have heard of you elsewhere. Do you
not bring men to the gallows for a living?''

``Aye, but only outlaws such as Robin Hood.''

``But pray what harm has Robin Hood done you?''

``He is a highway robber,'' said Sir Guy, evading the question.

``Has he ever taken from the rich that he did not give again to the
poor? Does he not protect the women and children and side with weak and
helpless? Is not his greatest crime the shooting of a few King's deer?''

``Have done with your sophistry,'' said Sir Guy impatiently. ``I am more
than ever of opinion that you are one of Robin's men yourself.''

``I have told you I am not,'' quoth Robin briefly. ``But if I am to help
you catch him, what is your plan?''

``Do you see this silver bugle?'' said the other. ``A long blast upon it
will summon the Sheriff and all his men, when once I have Robin within
my grasp. And if you show him to me, I'll give you the half of my forty
pounds reward.''

``I would not help hang a man for ten times forty pounds,'' said the
outlaw. ``Yet will I point out Robin to you for the reward I find at my
sword's point. I myself am Robin Hood of Sherwood and Barnesdale.''

``Then have at you!'' cried the other springing swiftly into action. His
sword leaped forth from beneath the horse's hide with the speed born of
long practice, and before Robin had come to guard, the other had smitten
at him full and foul. Robin eluded the lunge and drew his own weapon.

``A scurvy trick!'' quoth he grimly, ``to strike at a man unprepared.''

Then neither spoke more, but fell sternly to work--lunge and thrust and
ward and parry--for two full hours the weapons smote together sullenly,
and neither Robin Hood nor Sir Guy would yield an inch. I promise you
that if you could have looked forth on the fight from behind the trunk
of some friendly tree, you would have seen deadly sport such as few men
beheld in Sherwood Forest. For the fighters glared sullenly at each
other, the fires of hatred burning in their eyes. One was fighting for
his life; the other for a reward and the King's favor.

Still circled the bright blades swiftly in the air--now gleaming in the
peaceful sunlight--again hissing like maddened serpents. Neither had yet
touched the other, until Robin, in an unlucky moment, stumbled over the
projecting root of a tree; when Sir Guy, instead of giving him the
chance to recover himself, as any courteous knight would have done,
struck quickly at the falling man and wounded him in the left side.

``Ah, dear Lady in Heaven,'' gasped Robin uttering his favorite prayer,
``shield me now! `Twas never a man's destiny to die before his day.''

And adroitly he sprang up again, and came straight at the other with an
awkward but unexpected stroke. The knight had raised his weapon high to
give a final blow, when Robin reached beneath and across his guard. One
swift lunge, and Sir Guy of Gisborne staggered backward with a deep
groan, Robin's sword through his throat.

Robin looked at the slain man regretfully.

``You did bring it upon yourself,'' said he; ``and traitor and hireling
though you were, I would not willingly have killed you.''

He looked to his own wound. It was not serious, and he soon staunched
the blood and bound up the cut. Then he dragged the dead body into the
bushes, and took off the horse's hide and put it upon himself. He placed
his own cloak upon Sir Guy, and marked his face so none might tell who
had been slain. Robin's own figure and face were not unlike the other's.

Pulling the capul-hide well over himself, so that the helmet hid most of
his face, Robin seized the silver bugle and blew a long blast. It was
the blast that saved the life of Little John, over in Barnesdale, for
you and I have already seen how it caused the fond Sheriff to prick up
his ears and stay the hanging, and go scurrying up over the hill and
into the wood with his men in search of another victim.

In five-and-twenty minutes up came running a score of the Sheriff's best
archers.

``Did you signal us, lording?'' they asked, approaching Robin.

``Aye,'' said he, going to meet the puffing Sheriff.

``What news, what news, Sir Guy?'' said that officer.

``Robin Hood and Guy of Gisborne had a fight; and he that wears Robin's
cloak lies under the covert yonder.''

``The best news I have heard in all my life!'' exclaimed the Sheriff
rubbing his hands. ``I would that we could have saved him for the
hanging--though I cannot now complain.''

``The hanging?'' repeated Robin.

``Yes. This is our lucky day on the calendar. After you left me we
narrowly missed running one of the fellows--I believe `twas Will
Scarlet--to earth; and another who came to his relief we were just about
to hang, when your horn blew.''

``Who was the other?'' asked the disguised outlaw.

``Whom do you suppose?'' laughed the Sheriff. ``The best man in the
greenwood, next to Robin Hood himself--Little John, Reynold Greenleaf!''
For the Sheriff could not forget the name Little John had borne under
his own roof at Nottingham.

``Little John!'' thought Robin with a start. Verily that was a lucky
blast of the bugle! ``But I see you have not escaped without a
scratch,'' continued the Sheriff, becoming talkative through pure glee.
``Here, one of you men! Give Sir Guy of Gisborne your horse; while
others of you bury that dog of an outlaw where he lies. And let us
hasten back to Barnesdale and finish hanging the other.''

So they put spurs to their horses, and as they rode Robin forced himself
to talk merrily, while all the time he was planning the best way to
succor Little John.

``A boon, Sheriff,'' he said as they reached the gates of the town.

``What is it, worthy sir? You have but to speak.''

``I do not want any of your gold, for I have had a brave fight. But now
that I have slain the master, let me put an end to the man; so it shall
be said that Guy of Gisborne despatched the two greatest outlaws of
England in one day.''

``Have it as you will,'' said the Sheriff, ``but you should have asked a
knight's fee and double your reward, and it would have been yours. It
isn't every man that can take Robin Hood.'' ``No, Excellency,'' answered
Robin. ``I say it without boasting, that no man took Robin Hood
yesterday and none shall take him to-morrow.''

Then he approached Little John, who was still tied to the gallows-tree;
and he said to the Sheriff's men, ``Now stand you back here till I see
if the prisoner has been shrived.'' And he stooped swiftly, and cut
Little John's bonds, and thrust into his hands Sir Guy's bow and arrows,
which he had been careful to take.

```Tis I, Robin!'' he whispered. But in truth, Little John knew it
already, and had decided there was to be no hanging that day.

Then Robin blew three loud blasts upon his own horn, and drew forth his
own bow; and before the astonished Sheriff and his men could come to
arms the arrows were whistling in their midst in no uncertain fashion.

And look! Through the gates and over the walls came pouring another
flight of arrows! Will Scarlet and Will Stutely had watched and planned
a rescue ever since the Sheriff and Robin rode back down the hill. Now
in good time they came; and the Sheriff's demoralized force turned tail
and ran, while Robin and Little John stood under the harmless gallows,
and sped swift arrows after them, and laughed to see them go.

Then they joined their comrades and hasted back to the good greenwood,
and there rested. They had got enough sport for one day.
