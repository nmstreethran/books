\chapter{How Little John Entered the Sheriff's Service}

\begin{quote}
List and hearken, gentlemen,\\
All ye that now be here,\\
Of Little John, that was Knight's-man,\\
Good mirth ye now shall hear.
\end{quote}

\lettrine{I}{t} had come around another Fair day at Nottingham town, and
folk crowded there by all the gates. Goods of many kinds were displayed
in gaily colored booths, and at every cross-street a free show was in
progress. Here and there, stages had been erected for the play at
quarter-staff, a highly popular sport.

There was a fellow, one Eric of Lincoln, who was thought to be the
finest man with the staff for miles around. His feats were sung about in
ballads through all the shire. A great boaster was he withal, and to-day
he strutted about on one of these corner stages, and vaunted of his
prowess, and offered to crack any man's crown for a shilling. Several
had tried their skill with Eric, but he had soon sent them spinning in
no gentle manner, amid the jeers and laughter of the onlookers.

A beggar-man sat over against Eric's stage and grinned every time a pate
was cracked. He was an uncouth fellow, ragged and dirty and unshaven.
Eric caught sight of his leering face at one of his boasts--for there
was a lull in the game, because no man else wanted to come within reach
of Eric's blows. Eric, I say, noticed the beggar-man grinning at him
rather impudently, and turned toward him sharply.

``How now, you dirty villain!'' quoth he, ``mend your manners to your
betters, or, by our Lady, I'll dust your rags for you.''

The beggar-man still grinned. ``I am always ready to mend my manners to
my betters,'' said he, ``but I am afraid you cannot teach me any better
than you can dust my jacket.''

``Come up! Come up!'' roared the other, flourishing his staff.

``That will I,'' said the beggar, getting up slowly and with difficulty.
``It will pleasure me hugely to take a braggart down a notch, an some
good man will lend me a stout quarter-staff.''

At this a score of idlers reached him their staves--being ready enough
to see another man have his head cracked, even if they wished to save
their own--and he took the stoutest and heaviest of all. He made a sorry
enough figure as he climbed awkwardly upon the stage, but when he had
gained it, he towered full half a head above the other, for all his
awkwardness. Nathless, he held his stick so clumsily that the crowd
laughed in great glee.

Now each man took his place and looked the other up and down, watching
warily for an opening. Only a moment stood they thus, for Eric, intent
on teaching this rash beggar a lesson and sweeping him speedily off the
stage, launched forth boldly and gave the other a sounding crack on the
shoulder. The beggar danced about, and made as though he would drop his
staff from very pain, while the crowd roared and Eric raised himself for
another crushing blow. But just then the awkward beggar came to life.
Straightening himself like a flash, he dealt Eric a back-handed blow,
the like of which he had never before seen. Down went the boaster to the
floor with a sounding thump, and the fickle people yelled and laughed
themselves purple; for it was a new sight to see Eric of Lincoln eating
dust.

But he was up again almost as soon as he had fallen, and right quickly
retreated to his own ringside to gather his wits and watch for an
opening. He saw instantly that he had no easy antagonist, and he came in
cautiously this time.

And now those who stood around saw the merriest game of quarter-staff
that was ever played inside the walls of Nottingham town. Both men were
on their guard and fenced with fine skill, being well matched in
prowess. Again and again did Eric seek to force an opening under the
other's guard, and just as often were his blows parried. The beggar
stood sturdily in his tracks contenting himself with beating off the
attack. For a long time their blows met like the steady crackling of
some huge forest fire, and Eric strove to be wary, for he now knew that
the other had no mean wits or mettle. But he grew right mad at last, and
began to send down blows so fierce and fast that you would have sworn a
great hail-storm was pounding on the shingles over your head. Yet he
never so much as entered the tall beggar's guard.

Then at last the stranger saw his chance and changed his tune of
fighting. With one upward stroke he sent Eric's staff whirling through
the air. With another he tapped Eric on the head; and, with a third
broad swing, ere the other could recover himself, he swept him clear off
the stage, much as you would brush a fly off the window pane.

Now the people danced and shouted and made so much ado that the
shop-keepers left their stalls and others came running from every
direction. The victory of the queer beggar made him immensely popular.
Eric had been a great bully, and many had suffered defeat and insult at
his hands. So the ragged stranger found money and food and drink
everywhere at his disposal, and he feasted right comfortably till the
afternoon.

Then a long bow contest came on, and to it the beggar went with some of
his new friends. It was held in the same arena that Robin had formerly
entered; and again the Sheriff and lords and ladies graced the scene
with their presence, while the people crowded to their places.

When the archers had stepped forward, the herald rose and proclaimed the
rules of the game: how that each man should shoot three shots, and to
him who shot best the prize of a yoke of fat steers should belong. A
dozen keen-eyed bowmen were there, and among them some of the best
fellows in the Forester's and Sheriff's companies. Down at the end of
the line towered the tall beggar-man, who must needs twang a bow-string
with the best of them.

The Sheriff noted his queer figure and asked: ``Who is that ragged
fellow?''

``'Tis he that hath but now so soundly cracked the crown of Eric of
Lincoln,'' was the reply.

The shooting presently began, and the targets soon showed a fine
reckoning. Last of all came the beggar's turn.

``By your leave,'' he said loudly, ``I'd like it well to shoot with any
other man here present at a mark of my own placing.'' And he strode down
the lists with a slender peeled sapling which he stuck upright in the
ground. ``There,'' said he, ``is a right good mark. Will any man try
it?''

But not an archer would risk his reputation on so small a target.

Whereupon the beggar drew his bow with seeming carelessness and split
the wand with his shaft.

``Long live the beggar!'' yelled the bystanders.

The Sheriff swore a full great oath, and said: ``This man is the best
archer that ever yet I saw.'' And he beckoned to him, and asked him:
``How now, good fellow, what is your name, and in what country were you
born?''

``In Holderness I was born,'' the man replied; ``men call me Reynold
Greenleaf.''

``You are a sturdy fellow, Reynold Greenleaf, and deserve better apparel
than that you wear at present. Will you enter my service? I will give
you twenty marks a year, above your living, and three good suits of
clothes.''

``Three good suits, say you? Then right gladly will I enter your
service, for my back has been bare this many a long day.''

Then Reynold turned him about to the crowd and shouted: ``Hark ye, good
people, I have entered the Sheriff's service, and need not the yoke of
steers for prize. So take them for yourselves, to feast withal.''

At this the crowd shouted more merrily than ever, and threw their caps
high into the air. And none so popular a man had come to Nottingham town
in many a long day as this same Reynold Greenleaf.

Now you may have guessed, by this time, who Reynold Greenleaf really
was; so I shall tell you that he was none other than Little John. And
forth went he to the Sheriff's house, and entered his service. But it
was a sorry day for the Sheriff when he got his new man. For Little John
winked his shrewd eye and said softly to himself: ``By my faith, I shall
be the worst servant to him that ever yet had he!''

Two days passed by. Little John, it must be confessed, did not make a
good servant. He insisted upon eating the Sheriff's best bread and
drinking his best wine, so that the steward waxed wroth. Nathless the
Sheriff held him in high esteem, and made great talk of taking him along
on the next hunting trip.

It was now the day of the banquet to the butchers, about which we have
already heard. The banquet hall, you must know, was not in the main
house, but connected with it by a corridor. All the servants were
bustling about making preparations for the feast, save only Little John,
who must needs lie abed the greater part of the day. But he presented
himself at last, when the dinner was half over; and being desirous of
seeing the guests for himself he went into the hall with the other
servants to pass the wine. First, however, I am afraid that some of the
wine passed his own lips while he went down the corridor. When he
entered the banqueting hall, whom should he see but Robin Hood himself.
We can imagine the start of surprise felt by each of these bold fellows
upon seeing the other in such strange company. But they kept their
secrets, as we have seen, and arranged to meet each other that same
night. Meanwhile, the proud Sheriff little knew that he harbored the two
chief outlaws of the whole countryside beneath his roof.

After the feast was over and night was beginning to advance, Little John
felt faint of stomach and remembered him that he had eaten nothing all
that day. Back went he to the pantry to see what eatables were laid by.
But there, locking up the stores for the night, stood the fat steward.

``Good Sir Steward,'' said Little John, ``give me to dine, for it is
long for Greenleaf to be fasting.''

The steward looked grimly at him and rattled the keys at his girdle.

``Sirrah lie-abed,'' quoth he, ``'tis late in the day to be talking of
eating. Since you have waited thus long to be hungry, you can e'en take
your appetite back to bed again.''

``Now by mine appetite, that will I not do,'' cried Little John. ``Your
own paunch of fat would be enough for any bear to sleep on through the
winter. But my stomach craves food, and food it shall have!''

Saying this he brushed past the steward and tried the door, but it was
locked fast; whereat the fat steward chuckled and jangled his keys
again.

Then was Little John right mad, and he brought down his huge fist on the
door-panel with a sledge-hammer blow that shivered an opening you could
thrust your hand into. Little John stooped and peered through the hole
to see what food lay within reach, when crack! went the steward's keys
upon his crown, and the worthy danced around him playing a tattoo that
made Little John's ears ring. At this he turned upon the steward and
gave him such a rap that his back went nigh in two, and over went the
fat fellow rolling on the floor.

``Lie there,'' quoth Little John, ``till ye find strength to go to bed.
Meanwhile, I must be about my dinner.'' And he kicked open the buttery
door without ceremony and brought to light a venison pasty and cold
roast pheasant--goodly sights to a hungry man. Placing these down on a
convenient shelf he fell to with right good will. So Little John ate and
drank as much as he would.

Now the Sheriff had in his kitchen a cook, a stout man and bold, who
heard the rumpus and came in to see how the land lay. There sat Little
John eating away for dear life, while the fat steward was rolled under
the table like a bundle of rags.

``I make my vow!'' said the cook, ``you are a shrewd hind to dwell thus
in a household, and ask thus to dine.'' So saying he laid aside his spit
and drew a good sword that hung at his side.

``I make my vow!'' said Little John, ``you are a bold man and hardy to
come thus between me and my meat. So defend yourself and see that you
prove the better man.'' And he drew his own sword and crossed weapons
with the cook.

Then back and forth they clashed with sullen sound. The old ballad which
tells of their fight says that they thought nothing for to flee, but
stiffly for to stand. There they fought sore together, two miles away
and more, but neither might the other harm for the space of a full hour.

``I make my vow!'' cried Little John, ``you are the best swordsman that
ever yet I saw. What say you to resting a space and eating and drinking
good health with me. Then we may fall to again with the swords.''

``Agreed!'' said the cook, who loved good fare as well as a good fight;
and they both laid by their swords and fell to the food with hearty
will. The venison pasty soon disappeared, and the roast pheasant flew at
as lively a rate as ever the bird itself had sped. Then the warriors
rested a space and patted their stomachs, and smiled across at each
other like bosom friends; for a man when he as dined looks out
pleasantly upon the world.

``And now good Reynold Greenleaf,'' said the cook, ``we may as well
settle this brave fight we have in hand.''

``A true saying,'' rejoined the other, ``but first tell me, friend--for
I protest you are my friend henceforth--what is the score we have to
settle?''

``Naught save who can handle the sword best,'' said the cook. ``By my
troth I had thought to carve you like a capon ere now.''

``And I had long since thought to shave your ears,'' replied Little
John. ``This bout we can settle in right good time. But just now I and
my master have need of you, and you can turn your stout blade to better
service than that of the Sheriff.''

``Whose service would that be?'' asked the cook.

``Mine,'' answered a would-be butcher entering the room, ``and I am
Robin Hood.''
