\chapter{How King Richard Came to Sherwood Forest}

\begin{quote}
King Richard hearing of the pranks\\
Of Robin Hood and his men,\\
He much admired and more desired\\
To see both him and them.

Then Robin takes a can of ale:\\
“Come let us now begin;\\
And every man shall have his can;\\
Here’s a health unto the King!”
\end{quote}

\lettrine{F}{riar} Tuck had nursed Little John's wounded knee so
skilfully that it was now healed. In sooth, the last part of the nursing
depended more upon strength than skill; for it consisted chiefly of
holding down the patient, by main force, to his cot. Little John had felt
so well that he had insisted upon getting up before the wound was healed;
and he would have done so, if the friar had not piled some holy books
upon his legs and sat upon his stomach.

Under this vigorous treatment Little John was constrained to lie quiet
until the friar gave him leave to get up. At last he had this leave, and
he and the friar went forth to join the rest of the band, who were right
glad to see them, you may be sure. They sat around a big fire, for `twas
a chilly evening, and they feasted and made merry, in great content.

A cold rain set in, later, but the friar wended his way back, nathless,
to his little hermitage. There he made himself a cheerful blaze, and
changed his dripping robe, and had sat himself down, with a sigh of
satisfaction, before a tankard of hot mulled wine and a pasty, when
suddenly a voice was heard on the outside, demanding admission. His
kennel of dogs set up furious uproar, on the instant, by way of proving
the fact of a stranger's presence.

``Now by Saint Peter!'' growled the friar, ``who comes here at this
unseemly hour? Does he take this for a hostelry? Move on, friend, else
my mulled wine will get cold!''

So saying he put the tankard to his lips, when a thundering rap sounded
upon the door-panel, making it to quiver, and causing Tuck almost to
drop his tankard; while an angry voice shouted, ``Ho! Within there!
Open, I say!''

``Go your way in peace!'' roared back the friar; ``I can do nothing for
you. `Tis but a few miles to Gamewell, if you know the road.''

``But I do not know the road, and if I did I would not budge another
foot. `Tis wet without and dry within. So open, without further
parley!''

``A murrain seize you for disturbing a holy man in his prayers!''
muttered Tuck savagely. Nathless, he was fain to unbar the door in order
to keep it from being battered down. Then lighting a torch at his fire
and whistling for one of his dogs, he strode forth to see who his
visitor might be.

The figure of a tall knight clad in a black coat of mail, with plumed
helmet, stood before him. By his side stood his horse, also caparisoned
in rich armor.

``Have you no supper, brother?'' asked the Black Knight curtly. ``I must
beg of you a bed and a bit of roof, for this night, and fain would
refresh my body ere I sleep.''

``I have no room that even your steed would deign to accept, Sir Knight;
and naught save a crust of bread and pitcher of water.''

``I' faith, I can smell better fare than that, brother, and must e'en
force my company upon you, though I shall recompense it for gold in the
name of the church. As for my horse, let him but be blanketed and put on
the sheltered side of the house.''

And without further parley the knight boldly strode past Tuck and his
dog and entered the hermitage. Something about his masterful air pleased
Tuck, in spite of his churlishness.

``Sit you down, Sir Knight,'' quoth he, ``and I will fasten up up your
steed, and find him somewhat in the shape of grain. Half, also, of my
bed and board is yours, this night; but we shall see later who is the
better man, and is to give the orders!''

``With all my soul!'' said the knight, laughing. ``I can pay my keeping
in blows or gold as you prefer.''

The friar presently returned and drew up a small table near the fire.

``Now, Sir Knight,'' quoth he, ``put off your sword and helm and such
other war-gear as it pleases you, and help me lay this table, for I am
passing hungry.''

The knight did as he was told, and put aside the visor which had hid his
face. He was a bronzed and bearded man with blue eyes, and hair shot
with gold, haughty but handsome withal.

Then once again the priest sat him down to his pasty and mulled wine,
right hopefully. He spoke his grace with some haste, and was surprised
to hear his guest respond fittingly in the Latin tongue. Then they
attacked the wine and pasty valiantly, and the Black Knight made good
his word of being in need of refreshment. Tuck looked ruefully at the
rapidly disappearing food, but came to grudge it not, by reason of the
stories with which his guest enlivened the meal. The wine and warmth of
the room had cheered them both, and they were soon laughing uproariously
as the best of comrades in the world. The Black Knight, it seemed, had
traveled everywhere. He had been on crusades, had fought the courteous
Saladin, had been in prison, and often in peril. But now he spoke of it
lightly, and laughed it off, and made himself so friendly that Friar
Tuck was like to choke with merriment. So passed the time till late; and
the two fell asleep together, one on each side of the table which had
been cleared to the platters.

In the morning Friar Tuck awoke disposed to be surly, but was speedily
mollified by the sight of the Black Knight, who had already risen gay as
a lark, washed his face and hands, and was now stirring a hot gruel over
the fire.

``By my faith, I make a sorry host!'' cried Tuck springing to his feet.
And later as they sat at breakfast, he added, ``I want not your gold, of
which you spoke last night; but instead I will do what I can to speed
you on your way whenever you wish to depart.''

``Then tell me,'' said the knight, ``how I may find Robin Hood the
outlaw; for I have a message to him from the King. All day yesterday I
sought him, but found him not.''

Friar Tuck lifted up his hands in holy horror. ``I am a lover of peace,
Sir Knight, and do not consort with Robin's bold fellows.''

``Nay, I think no harm of Master Hood,'' said the knight; ``but much I
yearn to have speed with him in mine own person.''

``If that be all, mayhap I can guide you to his haunts,'' said Tuck, who
foresaw in this knight a possible gold-bag for Robin. ``In sooth, I
could not well live in these woods without hearing somewhat of the
outlaws; but matters of religion are my chief joy and occupation.''

``I will go with you, brother,'' said the Black Knight.

So without more ado they went their way into the forest, the knight
riding upon his charger, and Tuck pacing along demurely by his side.

The day had dawned clear and bright, and now with the sun a good three
hours high a sweet autumn fragrance was in the air. The wind had just
that touch of coolness in it which sets the hunter's blood to tingling;
and every creature of nature seemed bounding with joyous life.

The knight sniffed the fresh air in delight.

``By my halidom!'' quoth he; ``but the good greenwood is the best place
to live in, after all! What court or capital can equal this, for
full-blooded men?''

``None of this earth,'' replied Tuck smilingly. And once more his heart
warmed toward the courteous stranger.

They had not proceeded more than three or four miles along the way from
Fountain Abbey to Barnesdale, when of a sudden the bushes just ahead of
them parted and a well-knit man with curling brown hair stepped into the
road and laid his hand upon the knight's bridle.

It was Robin Hood. He had seen Friar Tuck, a little way back, and
shrewdly suspected his plan. Tuck, however, feigned not to know him at
all.

``Hold!'' cried Robin; ``I am in charge of the highway this day, and
must exact an accounting from all passersby.''

``Who is it bids me hold?'' asked the knight quietly. ``I am not i' the
habit of yielding to one man.''

``Then here are others to keep me company,'' said Robin clapping his
hands. And instantly a half-score other stalwart fellows came out of the
bushes and stood beside him.

``We be yeomen of the forest, Sir Knight,'' continued Robin, ``and live
under the greenwood tree. We have no means of support--thanks to the
tyranny of our over-lords--other than the aid which fat churchmen and
goodly knights like yourselves can give. And as ye have churches and
rents, both, and gold in great plenty, we beseech ye for Saint Charity
to give us some of your spending.''

``I am but a poor monk, good sir!'' said Friar Tuck in a whining voice,
``and am on my way to the shrine of Saint Dunstan, if your
worshipfulness will permit.''

``Tarry a space with us,'' answered Robin, biting back a smile, ``and we
will speed you on your way.''

The Black Knight now spoke again. ``But we are messengers of the King,''
quoth he; ``His Majesty himself tarries near here and would have speech
with Robin Hood.''

``God save the King!'' said Robin, doffing his cap loyally; ``and all
that wish him well! I am Robin Hood, but I say cursed be the man who
denies our liege King's sovereignty!''

``Have a care!'' said the knight, ``or you shall curse yourself!''

``Nay, not so,'' replied Robin curtly; ``the King has no more devoted
subject than I. Nor have I despoiled aught of his save, mayhap, a few
deer for my hunger. My chief war is against the clergy and barons of the
land who bear down upon the poor. But I am glad,'' he continued, ``that
I have met you here; and before we end you shall be my friend and taste
of our greenwood cheer.''

``But what is the reckoning?'' asked the knight. ``For I am told that
some of your feasts are costly.''

``Nay,'' responded Robin waving his hands, ``you are from the King.
Nathless--how much money is in your purse?''

``I have no more than forty gold pieces, seeing that I have lain a
fortnight at Nottingham with the King, and have spent some goodly
amounts upon other lordings,'' replied the knight.

Robin took the forty pounds and gravely counted it. One half he gave to
his men and bade them drink the King's health with it. The other half he
handed back to the knight.

``Sir,'' said he courteously, ``have this for your spending. If you lie
with kings and lordings overmuch, you are like to need it.''

``Gramercy!'' replied the other smiling. ``And now lead on to your
greenwood hostelry.''

So Robin went on the one side of the knight's steed, and Friar Tuck on
the other, and the men went before and behind till they came to the open
glade before the caves of Barnesdale. Then Robin drew forth his bugle
and winded the three signal blasts of the band. Soon there came a
company of yeomen with its leader, and another, and a third, and a
fourth, till there were sevenscore yeomen in sight. All were dressed in
new livery of Lincoln green, and carried new bows in their hands and
bright short swords at their belts. And every man bent his knee to Robin
Hood ere taking his place before the board, which was already set.

A handsome dark-haired page stood at Robin's right hand to pour his wine
and that of the knightly guest; while the knight marveled much at all he
saw, and said within himself:

``These men of Robin Hood's give him more obedience than my fellows give
to me.''

At the signal from Robin the dinner began. There was venison and fowl
and fish and wheaten cake and ale and red wine in great plenty, and
`twas a goodly sight to see the smiles upon the hungry yeomen's faces.

First they listened to an unctuous grace from Friar Tuck, and then Robin
lifted high a tankard of ale.

``Come, let us now begin,'' quoth he, ``and every man shall have his
can. In honor of our guest who comes with royal word, here's a health
unto the King!''

The guest responded heartily to this toast, and round about the board it
went, the men cheering noisily for King Richard!

After the feast was over, Robin turned to his guest and said, ``Now you
shall see what life we lead, so that you may report faithfully, for good
or bad, unto the King.''

So at a signal from him, the men rose up and smartly bent their bows for
practice, while the knight was greatly astonished at the smallness of
the their targets. A wand was set up, far down the glade, and thereon
was balanced a garland of roses. Whosoever failed to speed his shaft
through the garland, without knocking it off the wand, was to submit to
a buffet from the hand of Friar Tuck.

``Ho, ho!'' cried the knight, as his late traveling companion rose up
and bared his brawny arm ready for service; ``so you, my friend, are
Friar Tuck!''

``I have not gainsaid it,'' replied Tuck growling at having betrayed
himself. ``But chastisement is a rule of the church, and I am seeking
the good of these stray sheep.''

The knight said no more, though his eyes twinkled; and the shooting
began.

David of Doncaster shot first and landed safely through the rose
garland. Then came Allan-a-Dale and Little John and Stutely and Scarlet
and many of the rest, while the knight held his breath from very
amazement. Each fellow shot truly through the garland, until Middle the
tinker--not to be outdone--stepped up for a trial. But alas! while he
made a fair shot for a townsman, the arrow never came within a
hand-breath of the outer rim of the garland.

``Come hither, fellow,'' said Little John coaxingly. ``The priest would
bless thee with his open hand.''

Then because Middle made a wry face, as though he had already received
the buffet, and loitered in his steps, Arthur-a-Bland and Will Stutely
seized him by the arms and stood him before the friar. Tuck's big arm
flashed through the air--``whoof!'' and stopped suddenly against the
tinker's ear; while Middle himself went rolling over and over on the
grass. He was stopped by a small bush, and up he sat, thrusting his head
through it, rubbing his ear and blinking up at the sky as though the
stars had fallen and struck him. The yeomen roared with merriment, and
as for the knight, he laughed till the tears came out of his blue eyes
and rolled down his face.

After Middle's mishap, others of the band seemed to lose their balance,
and fared in the same fashion. The garland would topple over in a most
impish way at every breath, although the arrows went through it. So
Middle `gan to feel better when he saw this one and that one tumbling on
the sward.

At last came Robin's turn. He shot carefully, but as ill luck would have
it the shaft was ill-feathered and swerved sidewise so that it missed
the garland by full three fingers. Then a great roar went up from the
whole company; for `twas rare that they saw their leader miss his mark.
Robin flung his bow upon the ground from very vexation.

``A murrain take it!'' quoth he. ``The arrow was sadly winged. I felt
the poor feather upon it as it left my fingers!''

Then suddenly seizing his bow again, he sped three shafts as fast as he
could sent them, and every one went clean through the garland.

``By Saint George!'' muttered the knight. ``Never before saw I such
shooting in all Christendom!''

The band cheered heartily at these last shots; but Will Scarlet came up
gravely to Robin.

``Pretty shooting, master!'' quoth he, ``but `twill not save you from
paying for the bad arrow. So walk up and take your medicine!''

``Nay, that may not be!'' protested Robin. ``The good friar belongs to
my company and has no authority to lift hands against me. But you, Sir
Knight, stand as it were for the King. I pray you, serve out my blow.''

``Not so!'' said Friar Tuck. ``My son, you forget I stand for the
church, which is greater even than the King.''

``Not in merry England,'' said the knight in a deep voice. Then rising
to his feet, he added, ``I stand ready to serve you, Master Hood.''

``Now out upon ye for an upstart knight!'' cried Friar Tuck. ``I told
you last night, sirrah, that we should yet see who was the better man!
So we will e'en prove it now, and thus settle who is to pay Robin
Hood.''

``Good!'' said Robin, ``for I want not to start a dispute between church
and state.''

``Good!'' also said the knight. ```Tis an easy way to end prattling.
Come, friar, strike and ye dare. I will give you first blow.''

``You have the advantage of an iron pot on your head and gloves on your
hands,'' said the friar; ``but have at ye! Down you shall go, if you
were Goliath of Gath.''

Once more the priest's brawny arm flashed through the air, and struck
with a ``whoof!'' But to the amazement of all, the knight did not budge
from his tracks, though the upper half of his body swerved slightly to
ease the force of the blow. A loud shout burst from the yeomen at this,
for the friar's fist was proverbial, and few of those present had not
felt the force of it in times past.

``Now `tis my turn,'' said his antagonist coolly, casting aside his
gauntlet. And with one blow of his fist the knight sent the friar
spinning to the ground.

If there had been uproar and shouting before, it was as naught to the
noise which now broke forth. Every fellow held his sides or rolled upon
the ground from laughter; every fellow, save one, and that was Robin
Hood.

``Out of the frying-pan into the fire!'' thought he. ``I wish I had let
the friar box my ears, after all!''

Robin's plight did, indeed, seem a sorry one, before the steel muscles
of his stranger. But he was saved from a tumble heels over head by an
unlooked-for diversion. A horn winded in the glade, and a party of
knights were seen approaching.

``To your arms!'' cried Robin, hurriedly seizing his sword and bow.

```Tis Sir Richard of the Lea!'' cried another, as the troop came
nearer.

And so it was. Sir Richard spurred forward his horse and dashed up to
the camp while the outlaws stood at stiff attention. When he had come
near the spot where the Black Knight stood, he dismounted and knelt
before him.

``I trust Your Majesty has not needed our arms before,'' he said humbly.

``It is the King!'' cried Will Scarlet, falling upon his knees.

``The King!'' echoed Robin Hood after a moment of dumb wonderment; and
he and all his men bent reverently upon their knees, as one man.
