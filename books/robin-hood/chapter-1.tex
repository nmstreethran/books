\chapter{How Robin Hood Became an Outlaw}

\begin{quote}
List and hearken, gentlemen,\\
That be of free-born blood,\\
I shall you tell of a good yeoman,\\
His name was Robin Hood.

Robin was a proud outlaw,\\
While as he walked on the ground.\\
So courteous an outlaw as he was one\\
Was never none else found.
\end{quote}

\lettrine{I}{n} the days of good King Harry the Second of England--he of the warring
sons--there were certain forests in the north country set aside for the
King's hunting, and no man might shoot deer therein under penalty of
death. These forests were guarded by the King's Foresters, the chief of
whom, in each wood, was no mean man but equal in authority to the
Sheriff in his walled town, or even to my lord Bishop in his abbey.

One of the greatest of royal preserves was Sherwood and Barnesdale
forests near the two towns of Nottingham and Barnesdale. Here for some
years dwelt one Hugh Fitzooth as Head Forester, with his good wife and
son Robert. The boy had been born in Lockesley town--in the year 1160,
stern records say--and was often called Lockesley, or Rob of Lockesley.
He was a comely, well-knit stripling, and as soon as he was strong
enough to walk his chief delight was to go with his father into the
forest. As soon as his right arm received thew and sinew he learned to
draw the long bow and speed a true arrow. While on winter evenings his
greatest joy was to hear his father tell of bold Will o' the Green, the
outlaw, who for many summers defied the King's Foresters and feasted
with his men upon King's deer. And on other stormy days the boy learned
to whittle out a straight shaft for the long bow, and tip it with gray
goose feathers.

The fond mother sighed when she saw the boy's face light up at these
woodland tales. She was of gentle birth, and had hoped to see her son
famous at court or abbey. She taught him to read and to write, to doff
his cap without awkwardness and to answer directly and truthfully both
lord and peasant. But the boy, although he took kindly to these lessons
of breeding, was yet happiest when he had his beloved bow in hand and
strolled at will, listening to the murmur of the trees.

Two playmates had Rob in these gladsome early days. One was Will
Gamewell, his father's brother's son, who lived at Gamewell Lodge, hard
by Nottingham town. The other was Marian Fitzwalter, only child of the
Earl of Huntingdon. The castle of Huntingdon could be seen from the top
of one of the tall trees in Sherwood; and on more than one bright day
Rob's white signal from this tree told Marian that he awaited her there:
for you must know that Rob did not visit her at the castle. His father
and her father were enemies. Some people whispered that Hugh Fitzooth
was the rightful Earl of Huntingdon, but that he had been defrauded out
of his lands by Fitzwalter, who had won the King's favor by a crusade to
the Holy Land. But little cared Rob or Marian for this enmity, however
it had arisen. They knew that the great green--wood was open to them,
and that the wide, wide world was full of the scent of flowers and the
song of birds.

Days of youth speed all too swiftly, and troubled skies come all too
soon. Rob's father had two other enemies besides Fitzwalter, in the
persons of the lean Sheriff of Nottingham and the fat Bishop of
Hereford. These three enemies one day got possession of the King's ear
and whispered therein to such good--or evil--purpose that Hugh Fitzooth
was removed from his post of King's Forester. He and his wife and Rob,
then a youth of nineteen, were descended upon, during a cold winter's
evening, and dispossessed without warning. The Sheriff arrested the
Forester for treason--of which, poor man, he was as guiltless as you or
I--and carried him to Nottingham jail. Rob and his mother were sheltered
over night in the jail, also, but next morning were roughly bade to go
about their business. Thereupon they turned for succor to their only
kinsman, Squire George of Gamewell, who sheltered them in all kindness.

But the shock, and the winter night's journey, proved too much for Dame
Fitzooth. She had not been strong for some time before leaving the
forest. In less than two months she was no more. Rob felt as though his
heart was broken at this loss. But scarcely had the first spring flowers
begun to blossom upon her grave, when he met another crushing blow in
the loss of his father. That stern man had died in prison before his
accusers could agree upon the charges by which he was to be brought to
trial.

Two years passed by. Rob's cousin Will was away at school; and Marian's
father, who had learned of her friendship with Rob, had sent his
daughter to the court of Queen Eleanor. So these years were lonely ones
to the orphaned lad. The bluff old Squire was kind to him, but secretly
could make nothing of one who went about brooding and as though seeking
for something he had lost. The truth is that Rob missed his old life in
the forest no less than his mother's gentleness, and his father's
companionship. Every time he twanged the string of the long bow against
his shoulder and heard the gray goose shaft sing, it told him of happy
days that he could not recall.

One morning as Rob came in to breakfast, his uncle greeted him with, ``I
have news for you, Rob, my lad!'' and the hearty old Squire finished his
draught of ale and set his pewter tankard down with a crash.

``What may that be, Uncle Gamewell?'' asked the young man.

``Here is a chance to exercise your good long bow and win a pretty
prize. The Fair is on at Nottingham, and the Sheriff proclaims an
archer's tournament. The best fellows are to have places with the King's
Foresters, and the one who shoots straightest of all will win for prize
a golden arrow--a useless bauble enough, but just the thing for your
lady love, eh, Rob my boy?'' Here the Squire laughed and whacked the
table again with his tankard.

Rob's eyes sparkled. ```Twere indeed worth shooting for, uncle mine,''
he said. ``I should dearly love to let arrow fly alongside another man.
And a place among the Foresters is what I have long desired. Will you
let me try?''

``To be sure,'' rejoined his uncle. ``Well I know that your good mother
would have had me make a clerk of you; but well I see that the greenwood
is where you will pass your days. So, here's luck to you in the bout!''
And the huge tankard came a third time into play.

The young man thanked his uncle for his good wishes, and set about
making preparations for the journey. He traveled lightly; but his yew
bow must needs have a new string, and his cloth-yard arrows must be of
the straightest and soundest.

One fine morning, a few days after, Rob might have been seen passing by
way of Lockesley through Sherwood Forest to Nottingham town. Briskly
walked he and gaily, for his hopes were high and never an enemy had he
in the wide world. But `twas the very last morning in all his life when
he was to lack an enemy! For, as he went his way through Sherwood,
whistling a blithe tune, he came suddenly upon a group of Foresters,
making merry beneath the spreading branches of an oak-tree. They had a
huge meat pie before them and were washing down prodigious slices of it
with nut brown ale.

One glance at the leader and Rob knew at once that he had found an
enemy. `Twas the man who had usurped his father's place as Head
Forester, and who had roughly turned his mother out in the snow. But
never a word said he for good or bad, and would have passed on his way,
had not this man, clearing his throat with a huge gulp, bellowed out:
``By my troth, here is a pretty little archer! Where go you, my lad,
with that tupenny bow and toy arrows? Belike he would shoot at
Nottingham Fair! Ho! Ho!''

A roar of laughter greeted this sally. Rob flushed, for he was mightily
proud of his shooting.

``My bow is as good as yours,'' he retorted, ``and my shafts will carry
as straight and as far. So I'll not take lessons of any of ye.''

They laughed again loudly at this, and the leader said with frown:

``Show us some of your skill, and if you can hit the mark here's twenty
silver pennies for you. But if you hit it not you are in for a sound
drubbing for your pertness.''

``Pick your own target,'' quoth Rob in a fine rage. ``I'll lay my head
against that purse that I can hit it.''

``It shall be as you say,'' retorted the Forester angrily, ``your head
for your sauciness that you hit not my target.''

Now at a little rise in the wood a herd of deer came grazing by, distant
full fivescore yards. They were King's deer, but at that distance seemed
safe from any harm. The Head Forester pointed to them.

``If your young arm could speed a shaft for half that distance, I'd
shoot with you.''

``Done!'' cried Rob. ``My head against twenty pennies I'll cause yon
fine fellow in the lead of them to breathe his last.''

And without more ado he tried the string of his long bow, placed a shaft
thereon, and drew it to his ear. A moment, and the quivering string sang
death as the shaft whistled across the glade. Another moment and the
leader of the herd leaped high in his tracks and fell prone, dyeing the
sward with his heart's blood.

A murmur of amazement swept through the Foresters, and then a growl of
rage. He that had wagered was angriest of all.

``Know you what you have done, rash youth?'' he said. ``You have killed
a King's deer, and by the laws of King Harry your head remains forfeit.
Talk not to me of pennies but get ye gone straight, and let me not look
upon your face again.''

Rob's blood boiled within him, and he uttered a rash speech. ``I have
looked upon your face once too often already, my fine Forester. `Tis you
who wear my father's shoes.''

And with this he turned upon his heel and strode away.

The Forester heard his parting thrust with an oath. Red with rage he
seized his bow, strung an arrow, and without warning launched it full
af' Rob. Well was it for the latter that the Forester's foot turned on a
twig at the critical instant, for as it was the arrow whizzed by his ear
so close as to take a stray strand of his hair with it. Rob turned upon
his assailant, now twoscore yards away.

``Ha!'' said he. ``You shoot not so straight as I, for all your bravado.
Take this from the tupenny bow!''

Straight flew his answering shaft. The Head Forester gave one cry, then
fell face downward and lay still. His life had avenged Rob's father, but
the son was outlawed. Forward he ran through the forest, before the band
could gather their scattered wits--still forward into the great
greenwood. The swaying trees seemed to open their arms to the wanderer,
and to welcome him home.

Toward the close of the same day, Rob paused hungry and weary at the
cottage of a poor widow who dwelt upon the outskirts of the forest. Now
this widow had often greeted him kindly in his boyhood days, giving him
to eat and drink. So he boldly entered her door. The old dame was right
glad to see him, and baked him cakes in the ashes, and had him rest and
tell her his story. Then she shook her head.

```Tis an evil wind that blows through Sherwood,'' she said. ``The poor
are despoiled and the rich ride over their bodies. My three sons have
been outlawed for shooting King's deer to keep us from starving, and now
hide in the wood. And they tell me that twoscore of as good men as ever
drew bow are in hiding with them.''

``Where are they, good mother?'' cried Rob. ``By my faith, I will join
them.''

``Nay, nay,'' replied the old woman at first. But when she saw that
there was no other way, she said: ``My sons will visit me to-night. Stay
you here and see them if you must.''

So Rob stayed willingly to see the widow's sons that night, for they
were men after his own heart. And when they found that his mood was with
them, they made him swear an oath of fealty, and told him the haunt of
the band--a place he knew right well. Finally one of them said:

``But the band lacks a leader--one who can use his head as well as his
hand. So we have agreed that he who has skill enough to go to
Nottingham, an outlaw, and win the prize at archery, shall be our
chief.''

Rob sprang to his feet. ``Said in good time!'' cried he, ``for I had
started to that self-same Fair, and all the Foresters, and all the
Sheriff's men in Christendom shall not stand between me and the center
of their target!''

And though he was but barely grown he stood so straight and his eye
flashed with such fire that the three brothers seized his hand and
shouted:

``A Lockesley! a Lockesley! if you win the golden arrow you shall be
chief of outlaws in Sherwood Forest!''

So Rob fell to planning how he could disguise himself to go to
Nottingham town; for he knew that the Foresters had even then set a
price on his head in the market-place.

It was even as Rob had surmised. The Sheriff of Nottingham posted a
reward of two hundred pounds for the capture, dead or alive, of one
Robert Fitzooth, outlaw. And the crowds thronging the streets upon that
busy Fair day often paused to read the notice and talk together about
the death of the Head Forester.

But what with wrestling bouts and bouts with quarter-staves, and
wandering minstrels, there came up so many other things to talk about,
that the reward was forgotten for the nonce, and only the Foresters and
Sheriff's men watched the gates with diligence, the Sheriff indeed
spurring them to effort by offers of largess. His hatred of the father
had descended to the son.

The great event of the day came in the afternoon. It was the archer's
contest for the golden arrow, and twenty men stepped forth to shoot.
Among them was a beggar-man, a sorry looking fellow with leggings of
different colors, and brown scratched face and hands. Over a tawny shock
of hair he had a hood drawn, much like that of a monk. Slowly he limped
to his place in the line, while the mob shouted in derision. But the
contest was open to all comers, so no man said him nay.

Side by side with Rob--for it was he--stood a muscular fellow of swarthy
visage and with one eye hid by a green bandage. Him also the crowd
jeered, but he passed them by with indifference while he tried his bow
with practiced hand.

A great crowd had assembled in the amphitheater enclosing the lists. All
the gentry and populace of the surrounding country were gathered there
in eager expectancy. The central box contained the lean but pompous
Sheriff, his bejeweled wife, and their daughter, a supercilious young
woman enough, who, it was openly hinted, was hoping to receive the
golden arrow from the victor and thus be crowned queen of the day.

Next to the Sheriff's box was one occupied by the fat Bishop of
Hereford; while in the other side was a box wherein sat a girl whose
dark hair, dark eyes, and fair features caused Rob's heart to leap.
`Twas Maid Marian! She had come up for a visit from the Queen's court at
London town, and now sat demurely by her father the Earl of Huntingdon.
If Rob had been grimly resolved to win the arrow before, the sight of
her sweet face multiplied his determination an hundredfold. He felt his
muscles tightening into bands of steel, tense and true. Yet withal his
heart would throb, making him quake in a most unaccountable way.

Then the trumpet sounded, and the crowd became silent while the herald
announced the terms of the contest. The lists were open to all comers.
The first target was to be placed at thirty ells distance, and all those
who hit its center were allowed to shoot at the second target, placed
ten ells farther off. The third target was to be removed yet farther,
until the winner was proved. The winner was to receive the golden arrow,
and a place with the King's Foresters. He it was also who crowned the
queen of the day.

The trumpet sounded again, and the archers prepared to shoot. Rob looked
to his string, while the crowd smiled and whispered at the odd figure he
cut, with his vari-colored legs and little cape. But as the first man
shot, they grew silent.

The target was not so far but that twelve out of the twenty contestants
reached its inner circle. Rob shot sixth in the line and landed fairly,
being rewarded by an approving grunt from the man with the green
blinder, who shot seventh, and with apparent carelessness, yet true to
the bull's-eye.

The mob cheered and yelled themselves hoarse at this even marksmanship.
The trumpet sounded again, and a new target was set up at forty ells.

The first three archers again struck true, amid the loud applause of the
onlookers; for they were general favorites and expected to win. Indeed
`twas whispered that each was backed by one of the three dignitaries of
the day. The fourth and fifth archers barely grazed the center. Rob
fitted his arrow quietly and with some confidence sped it unerringly
toward the shining circle.

``The beggar! the beggar!'' yelled the crowd; ``another bull for the
beggar!'' In truth his shaft was nearer the center than any of the
others. But it was not so near that ``Blinder,'' as the mob had promptly
christened his neighbor, did not place his shaft just within the mark.
Again the crowd cheered wildly. Such shooting as this was not seen every
day in Nottingham town.

The other archers in this round were disconcerted by the preceding
shots, or unable to keep the pace. They missed one after another and
dropped moodily back, while the trumpet sounded for the third round, and
the target was set up fifty ells distant.

``By my halidom you draw a good bow, young master,'' said Rob's queer
comrade to him in the interval allowed for rest. ``Do you wish me to
shoot first on this trial?''

``Nay,'' said Rob, ``but you are a good fellow by this token, and if I
win not, I hope you may keep the prize from yon strutters.'' And he
nodded scornfully to the three other archers who were surrounded by
their admirers, and were being made much of by retainers of the Sheriff,
the Bishop, and the Earl. From them his eye wandered toward Maid
Marian's booth. She had been watching him, it seemed, for their eyes
met; then hers were hastily averted.

``Blinder's'' quick eye followed those of Rob. ``A fair maid, that,'' he
said smilingly, ``and one more worthy the golden arrow than the
Sheriff's haughty miss.''

Rob looked at him swiftly, and saw naught but kindliness in his glance.

``You are a shrewd fellow and I like you well,'' was his only comment.

Now the archers prepared to shoot again, each with some little care. The
target seemed hardly larger than the inner ring had looked, at the first
trial. The first three sped their shafts, and while they were fair shots
they did not more than graze the inner circle.

Rob took his stand with some misgiving. Some flecking clouds overhead
made the light uncertain, and a handful of wind frolicked across the
range in a way quite disturbing to a bowman's nerves. His eyes wandered
for a brief moment to the box wherein sat the dark-eyed girl. His heart
leaped! she met his glance and smiled at him reassuringly. And in that
moment he felt that she knew him despite his disguise and looked to him
to keep the honor of old Sherwood. He drew his bow firmly and, taking
advantage of a momentary lull in the breeze, launched the arrow straight
and true-singing across the range to the center of the target.

``The beggar! the beggar! a bull! a bull!'' yelled the fickle mob, who
from jeering him were now his warm friends. ``Can you beat that,
Blinder?''

The last archer smiled scornfully and made ready. He drew his bow with
ease and grace and, without seeming to study the course, released the
winged arrow. Forward it leaped toward the target, and all eyes followed
its flight. A loud uproar broke forth when it alighted, just without the
center and grazing the shaft sent by Rob. The stranger made a gesture of
surprise when his own eyes announced the result to him, but saw his
error. He had not allowed for the fickle gust of wind which seized the
arrow and carried it to one side. But for all that he was the first to
congratulate the victor.

``I hope we may shoot again,'' quoth he. ``In truth I care not for the
golden bauble and wished to win it in despite of the Sheriff for whom I
have no love. Now crown the lady of your choice.'' And turning suddenly
he was lost in the crowd, before Rob could utter what it was upon his
lips to say, that he would shoot again with him.

And now the herald summoned Rob to the Sheriff's box to receive the
prize.

``You are a curious fellow enough,'' said the Sheriff, biting his lip
coldly; ``yet you shoot well. What name go you by?''

Marian sat near and was listening intently.

``I am called Rob the Stroller, my Lord Sheriff,'' said the archer.

Marian leaned back and smiled.

``Well, Rob the Stroller, with a little attention to your skin and
clothes you would not be so bad a man,'' said the Sheriff. ``How like
you the idea of entering my service.

``Rob the Stroller has ever been a free man, my Lord, and desires no
service.''

The Sheriff's brow darkened, yet for the sake of his daughter and the
golden arrow, he dissembled.

``Rob the Stroller,'' said he, ``here is the golden arrow which has been
offered to the best of archers this day. You are awarded the prize. See
that you bestow it worthily.''

At this point the herald nudged Rob and half inclined his head toward
the Sheriff's daughter, who sat with a thin smile upon her lips. But Rob
heeded him not. He took the arrow and strode to the next box where sat
Maid Marian.

``Lady,'' he said, ``pray accept this little pledge from a poor stroller
who would devote the best shafts in his quiver to serve you.''

``My thanks to you, Rob in the Hood,'' replied she with a roguish
twinkle in her eye; and she placed the gleaming arrow in her hair, while
the people shouted, ``The Queen! the Queen!''

The Sheriff glowered furiously upon this ragged archer who had refused
his service, taken his prize without a word of thanks, and snubbed his
daughter. He would have spoken, but his proud daughter restrained him.
He called to his guard and bade them watch the beggar. But Rob had
already turned swiftly, lost himself in the throng, and headed straight
for the town gate.

That same evening within a forest glade a group of men--some twoscore
clad in Lincoln green--sat round a fire roasting venison and making
merry. Suddenly a twig crackled and they sprang to their feet and seized
their weapons.

``I look for the widow's sons,'' a clear voice said, ``and I come
alone.''

Instantly the three men stepped forward.

``Tis Rob!'' they cried; ``welcome to Sherwood Forest, Rob!'' And all
the men came and greeted him; for they had heard his story.

Then one of the widow's sons, Stout Will, stepped forth and said:

``Comrades all, ye know that our band has sadly lacked a leader--one of
birth, breeding, and skill. Belike we have found that leader in this
young man. And I and my brothers have told him that the band would
choose that one who should bring the Sheriff to shame this day and
capture his golden arrow. Is it not so?''

The band gave assent.

Will turned to Rob. ``What news bring you from Nottingham town?'' asked
he.

Rob laughed. ``In truth I brought the Sheriff to shame for mine own
pleasure, and won his golden arrow to boot. But as to the prize ye must
e'en take my word, for I bestowed it upon a maid.''

And seeing the men stood in doubt at this, he continued: ``But I'll
gladly join your band, and you take me, as a common archer. For there
are others older and mayhap more skilled than I.''

Then stepped one forward from the rest, a tall swarthy man. And Rob
recognized him as the man with the green blinder; only this was now
removed, and his freed eye gleamed as stoutly as the other one.

``Rob in the Hood--for such the lady called you,'' said he, ``I can
vouch for your tale. You shamed the Sheriff e'en as I had hoped to do;
and we can forego the golden arrow since it is in such fair hands. As to
your shooting and mine, we must let future days decide. But here I, Will
Stutely, declare that I will serve none other chief save only you.''

Then good Will Stutely told the outlaws of Rob's deeds, and gave him his
hand of fealty. And the widow's sons did likewise, and the other members
every one, right gladly; because Will Stutely had heretofore been the
truest bow in all the company. And they toasted him in nut brown ale,
and hailed him as their leader, by the name of Robin Hood. And he
accepted that name because Maid Marian had said it.

By the light of the camp-fire the band exchanged signs and passwords.
They gave Robin Hood a horn upon which he was to blow to summon them.
They swore, also, that while they might take money and goods from the
unjust rich, they would aid and befriend the poor and the helpless; and
that they would harm no woman, be she maid, wife, or widow. They swore
all this with solemn oaths, while they feasted about the ruddy blaze,
under the greenwood tree.

And that is how Robin Hood became an outlaw.
